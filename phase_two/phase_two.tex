\documentclass[letterpaper,graphics,11pt]{article}

%%%%%%%%%%%%%%%%%%%%%%%%%%%%%%%%%%%%%%%%%%%%%%%%%%%%%%%%%%%%%%%%%%%%%%%%%%%%%%%%

\usepackage[hmargin=2cm,vmargin=3cm]{geometry}
\usepackage{latexsym,amsmath,amssymb}
\usepackage{amsmath}
\usepackage{graphicx}
\usepackage{enumerate}
\usepackage{subfigure}
\usepackage[authoryear]{natbib}
\usepackage[small, compact]{titlesec}

%%%%%%%%%%%%%%%%%%%%%%%%%%%%%%%%%%%%%%%%%%%%%%%%%%%%%%%%%%%%%%%%%%%%%%%%%%%%%%%%

\title{600.415 - Project Phase 1}
\author{Adam Gerber \and Juri Ganitkevic}
\begin{document}

\maketitle

%%%%%%%%%%%%%%%%%%%%%%%%%%%%%%%%%%%%%%%%%%%%%%%%%%%%%%%%%%%%%%%%%%%%%%%%%%%%%%%%

\section*{}

\begin{enumerate}[(1)]

\item Adam Gerber and Juri Ganitkevic

\item In this project we will expand upon a novel optimization
  algorithm for multi-table intersections. The actual domain of the
  data and database design are therefore of secondary nature only. We
  are currently planning on using a sufficiently decomposed database
  of Twitter posts and user information to illustrate gains in join
  performance achieved by our algorithm.

\item Since we will not be using an estabished SQL database, but
  rather sample joins performed within our own code, our sample queries
  will be limited to natural joins over a large number of tables.

\item n/a

\item n/a

\item We currently plan on expanding upon an existing database of
  tweets by separating out currently textual features such as
  ``location'' into their own tables, thereby creating a larger number
  of tables for us to multi-join.
   
\item Our reports will be detailing the number and size of
  relations involved in the join as well as the methods used to speed
  up the muti-join.

\item We focus on join/intersection algorithms and optimization
  methods for those. A few approaches we have in mind are:
  \begin{itemize}
  \item Use a hiearchical key structure to efficiently perform
    intersections over $k$ ordered sets.
  \item Use Bloom filters to precisely and efficiently estimate
    overlap between key sets.
  \end{itemize}

\item Our software will be written in Java and should run on any
  recent machine without any additional libraries. Primarily, we will
  conduct (and can demonstrate) our experiments on our own desktop and
  laptop machines (i.e.\ up to dual-core i7 with 4GB RAM). We also
  intend to run a few large-scale experiments on the CLSP cluster
  machines with 128GB RAM.

\end{enumerate}

%%%%%%%%%%%%%%%%%%%%%%%%%%%%%%%%%%%%%%%%%%%%%%%%%%%%%%%%%%%%%%%%%%%%%%%%%%%%%%%%

\end{document}
